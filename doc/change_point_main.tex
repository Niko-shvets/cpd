     \documentclass[a4paper,12pt]{report} %размер бумаги устанавливаем А4, шрифт 
     
     \usepackage[intlimits]{amsmath}
\usepackage{amsthm,amsfonts}
\usepackage{amssymb}
\usepackage{mathrsfs}
%\usepackage{graphicx}
\usepackage[final]{graphicx,epsfig} 
\usepackage{longtable}
\usepackage{indentfirst}
\usepackage[utf8]{inputenc}

\usepackage[english]{babel}
\usepackage[usenames]{color}



\usepackage{hyperref}

\usepackage{easyeqn}
\usepackage{hyperref}

\renewcommand{\thesection}{\arabic{section}}
\renewcommand{\thesubsection}{\arabic{section}.\arabic{subsection}.}

%\usepackage[backend=bibtex]{biblatex}
%\bibliography{vldb}

\graphicspath{{img/}}
\newcommand{\imgh}[3]{\begin{figure}[!h]\center{\includegraphics[width=#1]{#2}}\caption{#3}\label{Fig:#2}\end{figure}}



    \usepackage{geometry} % Меняем поля страницы
    \geometry{left=2cm}% левое поле
    \geometry{right=1.5cm}% правое поле
    \geometry{top=1cm}% верхнее поле
    \geometry{bottom=2cm}% нижнее поле

 %   \renewcommand{\theenumi}{\arabic{enumi}}% Меняем везде перечисления на цифра.цифра
  %  \renewcommand{\labelenumi}{\arabic{enumi}}% Меняем везде перечисления на цифра.цифра
   % \renewcommand{\theenumii}{.\arabic{enumii}}% Меняем везде перечисления на цифра.цифра
   % \renewcommand{\labelenumii}{\arabic{enumi}.\arabic{enumii}.}% Меняем везде перечисления на цифра.цифра
  %  \renewcommand{\theenumiii}{.\arabic{enumiii}}% Меняем везде перечисления на цифра.цифра
   % \renewcommand{\labelenumiii}{\arabic{enumi}.\arabic{enumii}.\arabic{enumiii}.}% Меняем везде перечисления на цифра.цифра
    


 


\usepackage{verbatim}

\theoremstyle{definition}
%\newtheorem{problem}{\noindent\normalsize\bfЗадача \No\!\!}
\newtheorem{problem}{\noindent\normalsize\bf{}}[section]
\renewcommand{\theproblem}{\arabic{problem}}
\newtheorem*{example}{\noindent\normalsize\bf{}Пример}
\newtheorem*{definition}{\noindent\normalsize\bf{}Определение}

\newtheorem*{remark}{\noindent\normalsize\bf{}Remark}
\newtheorem{theorem}{\noindent\normalsize\bf{}Theorem}
\newtheorem*{lemma}{\noindent\normalsize\bf{}Lemma}
\newtheorem*{cons}{\noindent\normalsize\bf{}Consequence}


\newtheorem*{ordre}{\noindent\normalsize\bf{}Указание}

\newenvironment{solution}{\begin{proof}\vspace{1em}} {\end{proof} \vspace{2em}}


\newcommand{\rg}{\ensuremath{\mathrm{rg}}}
\newcommand{\grad}{\ensuremath{\mathrm{grad}}}
\newcommand{\diag}{\ensuremath{\mathrm{diag}}}
\newcommand{\const}{\ensuremath{\mathop{\mathrm{const}}}\nolimits}
\newcommand{\Var}{\ensuremath{\mathop{\mathbb{D}}}\nolimits}
\newcommand{\Exp}{\ensuremath{\mathrm{{\mathbb E}}}}
\newcommand{\E}{\ensuremath{\mathrm{{\mathbb E}}}}
\newcommand{\N}{\ensuremath{\mathrm{{\mathcal N}}}}
\newcommand{\PR}{\ensuremath{\mathrm{{\mathbb P}}}}
\newcommand{\Be}{\ensuremath{\mathrm{Be}}}
\newcommand{\Po}{\ensuremath{\mathrm{Po}}}
\newcommand{\Bi}{\ensuremath{\mathrm{Bi}}}
\newcommand{\Ker}{\ensuremath{\mathrm{Ker}}}
\newcommand{\Real}{\ensuremath{\mathrm{Re}}}
\newcommand{\Lin}{\ensuremath{\mathrm{Lin}}}
\newcommand{\Gl}{\ensuremath{\mathrm{Gl}}}
\newcommand{\mes}{\ensuremath{\mathrm{mes}}}
\newcommand{\cov}{\ensuremath{\mathrm{cov}}}
\newcommand{\fix}{\noindent\normalsize\frownie{} \color{red}} 


\usepackage{listings}

\usepackage{color}
\definecolor{gray}{rgb}{0.4,0.4,0.4}
\definecolor{darkblue}{rgb}{0.0,0.0,0.6}
\definecolor{cyan}{rgb}{0.0,0.6,0.6}

\lstset{
basicstyle=\ttfamily,
columns=fullflexible,
showstringspaces=false,
commentstyle=\color{gray}\upshape
}




\setcounter{tocdepth}{2}

  \begin{document}
    
    \selectlanguage{russian}

\begin{center}
\begin{huge}
\textbf{Обнаружение точки разладки при помощи обобщенного отношения правдоподобия \\}
\end{huge}
$ $ \\  
{\Large Бузун Назар postrealist@gmail.com }
\end{center}
   
    \tableofcontents 
    
  \section{График отношения правдоподобия}
  
  \subsection{Точка максимума правдоподобия}
  
  \[
  \alpha(\theta, \theta_0) = L(\theta) - L(\theta_0)   - (\theta - \theta_0)^T \nabla L_G( \theta_0) + \frac{1}{2} \Vert D (\theta - \theta_0) \Vert^2 
  \]
  
 Пусть в локальной области $\Theta_0(r) = \{\Vert D (\theta - \theta^*) \Vert < r \}$, где $\theta, \theta_0 \in \Theta_0(r)$, справедливо неравенство с вероятностью $1 - e^{-x}$ 
  
 \[\tag{a}
\frac{| \alpha(\theta, \theta_0)  |}{\Vert D(\theta - \theta_0) \Vert} \leq 2 \diamondsuit (r, x),  
  \]
 где $\diamondsuit (r, x) = (\delta (r) + 6 v_0 z_H(x) \omega ) r.$
    
Разделим выборку на два отрезка по индексу, правдоподобие и параметр части выборки слева будем индексировать  как $*_{1}$, справа  -- $*_{2}$. Выразим ОМП оценку на всей выборке через ОМП оценки слева ($\widehat{\theta}_1$) и справа ($\widehat{\theta}_2$):

\[
L(\theta) = L_1(\theta) + L_2(\theta) = L_1(\widehat{\theta}_1) + L_2(\widehat{\theta}_2) - 
\]
\[ -  \frac{1}{2} (\theta - \widehat{\theta}_1)^T D_1^2  (\theta - \widehat{\theta}_1)
-  \frac{1}{2} (\theta - \widehat{\theta}_2)^T D_2^2  (\theta - \widehat{\theta}_2) + O(r \diamondsuit (r, x)), 
\]     
\[
\frac{1}{2} (\theta - \widehat{\theta}_1)^T D_1^2  (\theta - \widehat{\theta}_1)
-  \frac{1}{2} (\theta - \widehat{\theta}_2)^T D_2^2  (\theta - \widehat{\theta}_2) 
 = 
\]
\[
 = \frac{1}{2} (\theta - \widehat{\theta})^T D^2  (\theta - \widehat{\theta}) + \text{const}(\theta),
\]
\[
\widehat{\theta} = D^{-2} (D_1^2 \widehat{\theta}_1 + D_2^2 \widehat{\theta}_2 ),
\quad D^2 = D_1^2 + D_2^2.
\]    

\begin{lemma}
Пусть $\tilde{\theta} = \arg\max L(\theta)$, $\widehat{\theta} \in \Theta_1(r) \cap \Theta_2(r)$, тогда с вероятностью $1 - 2 e^{-x}$
\[
\Vert D (\tilde{\theta} -  \widehat{\theta}) \Vert^2 \leq O (r \diamondsuit (r, x)).
\]
\end{lemma}

\begin{remark}
Для выполнения условия $\widehat{\theta} \in \Theta_1(r) \cap \Theta_2(r)$ требуется ограничение на изменение параметра $\theta^*$ вида
\[\tag{L*}
\Vert D(\theta_1^* - \theta_2^*) \Vert \leq r.
\]  
\end{remark}

\subsection{Отношение правдоподобия} 

\[
\triangle L =
L_1(\widehat{\theta}_1) + L_2(\widehat{\theta}_2) -  L(\widehat{\theta}) = 
\]
\[
=   \frac{1}{2} (\widehat{\theta} - \widehat{\theta}_1)^T D_1^2  (\widehat{\theta} - \widehat{\theta}_1)
+ \frac{1}{2} (\widehat{\theta} - \widehat{\theta}_2)^T D_2^2  (\widehat{\theta} - \widehat{\theta}_2) + O(r \diamondsuit (r, x)).
\]
\[
\widehat{\theta} - \widehat{\theta}_1 = D^{-2} (D_1^2 \widehat{\theta}_1 + D_2^2 \widehat{\theta}_2 ) - \widehat{\theta}_1 = D^{-2} D_2^2 ( \widehat{\theta}_2 -  \widehat{\theta}_1),
\]
\[
\widehat{\theta} - \widehat{\theta}_2 = D^{-2} (D_1^2 \widehat{\theta}_1 + D_2^2 \widehat{\theta}_2 ) - \widehat{\theta}_2 = D^{-2} D_1^2 ( \widehat{\theta}_1 -  \widehat{\theta}_2).
\]
\[
2 \triangle L =  ( \widehat{\theta}_2 -  \widehat{\theta}_1)^T \Sigma^2 ( \widehat{\theta}_2 -  \widehat{\theta}_1) + O(r \diamondsuit (r, x)),
\]
где 
\[\tag{Sigma}
\Sigma^2 = D_2^2 D^{-2} D_1^2 D^{-2} D_2^2 + D_1^2 D^{-2} D_2^2 D^{-2} D_1^2
 = D_1^2 D^{-2}D_2^2 \approx \frac{1}{4} D^2. 
\]

Предположим, что выполнены неравенства с вероятностью $1 - 2 e^{-x}$
\[
\Vert D_1(\widehat{\theta}_1 - \theta_1^*) - \xi_1 \Vert \leq \diamondsuit (r, x),
\]
\[
\Vert D_2(\widehat{\theta}_2 - \theta_2^*) - \xi_2 \Vert \leq \diamondsuit (r, x).
\]

Выполнив замены $\widehat{\theta}_2,  \widehat{\theta}_1$ в выражении для $\triangle L$ при помощи неравенства
\[
\vert \Vert a \Vert^2  - \Vert b \Vert^2 \vert \leq 2 \Vert a \Vert \Vert a - b \Vert + \Vert a - b \Vert^2,
\] 
приходим к следующему результату.

\begin{theorem}
\label{dl_theorem}
Пусть выполнено условие ($L^*$), а также условия квадратичной аппроксимации  Лапласа и с вероятностью $1 - 2 e^{-x}$ выполнено условие
\[
\Vert \xi_i \Vert \leq z(x), 
\quad z^2(x) = p + 6 \lambda_{\max} x.
\] 
Тогда в локальной области с вероятностью $1 - 8 e^{-x}$ справедливо равенство
\[
2 \triangle L = \Vert \triangle \xi_{12} + \triangle \theta_{12}^*\Vert^2  + O(\{r + z(x)\} f_{D} \diamondsuit (r, x)),
\]
где
\[
\triangle \xi_{12}  = \Sigma (D_2^{-1} \xi_2 - D_1^{-1} \xi_1),
\quad
\triangle \theta_{12}^*  = \Sigma (\theta_2^* - \theta_1^*), 
\]
\[
\left\Vert  \Sigma D^{-1}_i  \right\Vert_{\infty} \leq f_D, 
\quad i = \{1,2\}.
\]
\end{theorem}    

\begin{remark} При увеличении размера выборки $n \to \infty$ имеет место сходимость по распределению 
\[
\xi_{12} \to \mathcal{N}(0, I_p).
\]
\end{remark}

\begin{remark} 
Так как $\Vert \triangle \xi_{12} \Vert^2 \sim p$, то должно выполнятся
\[
r \diamondsuit (r, x) \sim \frac{r^2}{\sqrt{n}}  = o(p),
\] 
где $n$ -- длина выборки. Иначе Теорема \ref{dl_theorem} теряет смысл. Таким образом, из условия  ($L^*$) получаем, что разность параметра $\Vert \theta_2^* - \theta_1^* \Vert$ не должна превышать величины
\[
\frac{\sqrt{p}} {n^{1/4}},
\] 
так как $\Vert D \Vert  = O(\sqrt{n})$. Если же пренебречь членом $\Vert \triangle \xi_{12} \Vert^2 \sim p$, то приходим к условию
\[
\frac{r}{\sqrt{n}}  = o(\sqrt{p})
\Rightarrow 
\Vert \theta_2^* - \theta_1^* \Vert = o(\sqrt{p}).
\]
\end{remark}

Установим аналогичный результат (теорема \ref{dl_theorem}) для статистики $\sqrt{2 \triangle L}$. Из условия ($a$) получаем c вероятностью не менее $1 - 2e^{-x}$
\[
\left| \triangle L (\widehat{\theta}_1, \widehat{\theta}_2) - \frac{1}{2} \Vert \Sigma (\widehat{\theta}_2 -\widehat{\theta}_1) \Vert^2 \right| \leq 
2 \Vert D_1(\widehat{\theta}_1 - \widehat{\theta}) \Vert \diamondsuit (r, x) + 2 \Vert D_2(\widehat{\theta}_2 - \widehat{\theta}) \Vert \diamondsuit (r, x) 
\leq 
\] 
\[
\leq  4  \Vert  \Sigma(\widehat{\theta}_2 - \widehat{\theta}_1)  \Vert  \diamondsuit (r, x).
\]
Воспользуемся неравенством $|a - b| \leq |a^2 - b^2| / b, \; b >0$.
\[
\left| \sqrt{ 2\triangle L (\widehat{\theta}_1, \widehat{\theta}_2) } -  \Vert \Sigma (\widehat{\theta}_2 -\widehat{\theta}_1) \Vert \right| \leq 
8   \diamondsuit (r, x).
\]
Заменим $(\widehat{\theta}_1, \widehat{\theta}_2)$ на $(D_1^{-1}\xi_1 + \theta_1^*, \; D_2^{-1}\xi_2 + \theta_2^*)$
\[
\left| \Vert \Sigma (\widehat{\theta}_2 -\widehat{\theta}_1) \Vert  - 
\Vert \triangle \xi_{12} + \triangle \theta_{12}^* \Vert 
\right |
\leq  \Vert \Sigma(\widehat{\theta}_1 - \theta_1^*) - \Sigma D_1^{-1} \xi_1 \Vert
+ \Vert \Sigma(\widehat{\theta}_2 - \theta_2^*) - \Sigma D_2^{-1} \xi_2 \Vert
\leq  2 f_D \diamondsuit (r, x).
\]
Резюмируем результат 
\begin{theorem}
\label{dl_sq_theorem}

Пусть выполнено условие ($L^*$), а также условия квадратичной аппроксимации  Лапласа (a). Тогда с вероятностью не менее $1 - 4 e^{-x}$ в локальной области $\Theta_1(r) \cap \Theta_2(r)$ выполнено неравенство
\[
\left| 
\sqrt{ 2\triangle L (\widehat{\theta}_1, \widehat{\theta}_2) } - 
\Vert \triangle \xi_{12} + \triangle \theta_{12}^* \Vert 
\right| \leq 
(8 + 2 f_D)  \diamondsuit (r, x).
\]
где
\[
\triangle \xi_{12}  = \Sigma (D_2^{-1} \xi_2 - D_1^{-1} \xi_1),
\quad
\triangle \theta_{12}^*  = \Sigma (\theta_2^* - \theta_1^*), 
\]
\[
\left\Vert  \Sigma D^{-1}_i  \right\Vert_{\infty} \leq f_D, 
\quad i = \{1,2\}.
\]

\end{theorem}

\begin{remark}
Величина $f_D < 1$ на погрешность погрешность в теоремах \ref{dl_theorem},
\ref{dl_sq_theorem} не оказывает большого влияния. Однако, константу при $\diamondsuit (r, x)$ можно уменьшить,  раскладывая $L_1(\theta)$, $L_2(\theta)$ и $L(\theta)$ в окрестности точек  $\theta_1^*$, $\theta_2^*$ и $\theta^*$ вместо точек ОМП:

\[
2 \triangle L = - \Vert \xi \Vert^2 + \Vert \xi_1 \Vert^2 + \Vert \xi_2 \Vert^2   - 
2 \xi_1^T D_1 D^{-2} D_2^2 (\theta_2^* - \theta_1^*) + 2 \xi_2^T D_2 D^{-2} D_1^2 (\theta_2^* - \theta_1^*)+
\]
\[ + \Vert  D_1 D^{-2} D_2^2 (\theta_2^* - \theta_1^* )\Vert^2 + \Vert  D_2 D^{-2} D_1^2 (\theta_2^* - \theta_1^*) \Vert^2 \pm (2 \diamondsuit (r, x) r + 2 \delta(r) r^2) =
\]
\[
= - \Vert \xi \Vert^2 + \Vert \xi_1 \Vert^2 + \Vert \xi_2 \Vert^2   + 
2(D_2^{-1} \xi_2 - D_1^{-1} \xi_1)^T \Sigma^2 (\theta_2^* - \theta_1^*) + 
\Vert \Sigma (\theta_2^* - \theta_1^*) \Vert^2 \pm (2 \diamondsuit (r, x) r + 2 \delta(r) r^2).
\]
Заменим $\Vert \xi \Vert^2$ на $\Vert D^{-1}(D_1 \xi_1 + D_2 \xi_2) \Vert^2 \pm  2\diamondsuit (r, x) z(x) $. 
\[
- \Vert \xi \Vert^2 + \Vert \xi_1 \Vert^2 + \Vert \xi_2 \Vert^2  = \Vert \Sigma (D_2^{-1} \xi_2 - D_1^{-1} \xi_1) \Vert^2 \pm  2\diamondsuit (r, x) z(x).
\]
Приходим к конечному выражению
\[
\bigg |
2 \triangle L  -  \Vert \triangle \xi_{12} + \triangle \theta_{12}^* \Vert^2 
\bigg | 
\leq (4 \diamondsuit (r, x) r + 2 \delta(r) r^2).
\]
\end{remark}
    
 

\section{Бутстреп}

\begin{lemma}[Неравенство Хефдинга для векторов]
Пусть $\Exp X_i = 0$, $\Vert X_i \Vert < c_i$, тогда 
\[
\PR\left(\left\Vert \sum_{i=1}^{n} X_i \right\Vert > t \right) \leq \text{exp} \left( - \frac{(t - v)^2}{2v^2} \right), 
\quad t \geq v,
\]
\[
v^2 = \frac{1}{4} \sum_{i=1}^{n} c_i^2.
\]
\end{lemma}


\begin{suite}
\[
\PR\left(\left\Vert \sum_{i=1}^{n} X_i \right\Vert > v (1 + \sqrt{2x}) \right) \leq e^{-x}.
\]
\end{suite}


\begin{lemma}[Неравенство для отклонений нормы эмпирического процесса]
Пусть выполнены ограничения при $\Vert \gamma_1 \Vert = \Vert \gamma_2 \Vert = 1$
\[\tag{ED}
\log \Exp \text{exp} \left\{
\frac{\lambda}{\omega} \gamma_1 D_0^{-1}  \nabla^2 \zeta_i (\theta) D_0^{-1} \gamma_2 
\right\}
\leq \frac{\lambda^2 \nu_i^2}{2}, 
\quad \sum_i  \nu_i^2  = \nu_0^2.
\] 
Тогда при $\zeta = \sum_i \zeta_i$ в локальной области радиуса $r$ с вероятностью $1- e^{-x}$
\[
\Vert D_0^{-1} (\nabla \zeta (\theta_2) -  \nabla \zeta (\theta_1))  \Vert
\leq 12 \nu_0 z(x) \omega r.
\]
\end{lemma}
Рассмотрим функционал 
\[
\alpha^o(\theta_2, \theta_1) = L^{o}(\theta_2) - L^{o}(\theta_1) - (\theta_2 - \theta_1)^{T} \nabla  L^{o}(\theta_1) + \frac{1}{2} \Vert  D_0(\theta_2 - \theta_1) \Vert^2
\]
Будем подразумевать что дифференцирование производится по $\theta_2$ для функций типа $(\theta_2, \theta_1) \to \mathbb{R}$.
Оценим величину его среднего значения и отклонения от среднего 
\[
\Vert D_0^{-1} \nabla \Exp^{0} \alpha^o(\theta_2, \theta_1) \Vert =   \Vert D_0^{-1} \nabla  \alpha(\theta_2, \theta_1) \Vert 
\leq
2 \diamondsuit(r,x)
\]
c вероятностью $1 - e^{-x}$. 
\[
S(\theta_2, \theta_1) = D_0^{-1} \{\nabla \alpha^o(\theta_2, \theta_1) -  \Exp^{o} \nabla \alpha^o(\theta_2, \theta_1) \}
= \sum_{i=1}^{n} D_0^{-1} (\nabla l_i(\theta_2) - \nabla l_i(\theta_1))  (u_i-1),
\]
где $\Var u_i = 1$, $\Exp u_i = 1$. 
Обозначим за $ \overset{o}{S} = S(\theta_2, \theta_1) - \Exp S(\theta_2, \theta_1)$, тогда с вероятностью $1- e^{-x}$
\[
\Vert \overset{o}{S}  \Vert \leq 
12 \nu_0  z(x) \omega r \sqrt{\sum_{i=1}^{n} (\nu_i^2 / \nu_0^2) (u_i - 1)^2 } .
\] 
Ограничим $u_i$:
\[
\Exp e^{(u_i - 1)^2 / \sigma_u^2} \leq e,
\] 
тогда вероятностью $1 - e^{-t}$ 
\[
\sqrt{\sum_{i=1}^{n} (\nu_i^2 / \nu_0^2) (u_i - 1)^2 } \leq \sigma_u \sqrt{1+t} 
\]
Таким образом, с  вероятностью $1- 2e^{-x}$
\[
\Vert \overset{o}{S}  \Vert \leq 
12 \nu_0  z(x) \omega r \sigma_u \sqrt{1+x}. 
\]
Наложим ограничение на  $\Exp S(\theta_2, \theta_1)$ в виде
\[\tag{Lm}
\Vert D_0^{-1} \nabla^2 \Exp l_i (\theta)  D_0^{-1} \Vert \leq
\frac{C_i(r)}{n},
\]
в результате чего при помощи векторного неравенства Хефдинга получим с вероятностью $1 - e^{-x}$
\[
\Vert \Exp S(\theta_2, \theta_1) \Vert \leq 
\frac{1 + \sqrt{2x}}{2n}  C(r) r c_u,
\] 
где $C(r) = \sqrt{\sum_i C_i^2 (r)}$, $c_u = \max (u_i - 1)$. В итоге приходим к следующему результату 

\begin{theorem}[Бутстреп, Вилкс]
\label{wilks_boot}
При выполнении условий $(ED)$ и $(Lm)$, $(a)$, а также ограничений для весов $u_i$, в локальной области радиуса $r$ имеет место неравенство с вероятностью $1 - 4 e^{-x}$
\[\tag{a'}
| \alpha^o(\theta_2, \theta_1) |
\leq 
\Vert D_0 (\theta_2 - \theta_1)  \Vert
\left(
2 \diamondsuit(r,x) + 12 \nu_0  z(x) \omega r \sigma_u \sqrt{1+x} + 
\frac{1 + \sqrt{2x}}{2n}  C(r) r c_u
\right)  = 
\]
\[
 = \Vert D_0 (\theta_2 - \theta_1)  \Vert \; 2 \diamondsuit^{o}(r,x). 
\] 
\end{theorem}

Для вывода теоремы Фишера для взвешенной функции правдоподобия в мире бутстрепа введем переменную
\[
\chi^{o}(\theta_2, \theta_1) =  D_0^{-1} ( \nabla L^{o}(\theta_2) -   \nabla L^{o}(\theta_1) ) + D_0 (\theta_2-  \theta_1),
\] 
из верхней оценки которой легко получить теорему Фишера. Заметим, что
\[
\chi^{o}(\theta_2, \theta_1) = D_0^{-1} \nabla \alpha^{o}(\theta_2, \theta_1). 
\]
Тогда, проводя рассуждения аналогичные доказательству теоремы Вилкса, получаем верхнюю оценку $\chi^{o}(\theta_2, \theta_1)$.

\begin{theorem}[Бутстреп, Фишер]
\label{fisher_boot}
При выполнении условий $(ED)$ и $(Lm)$, $(a)$, а также ограничений для весов $u_i$, в локальной области радиуса $r$ имеет место неравенство с вероятностью $1 - 4 e^{-x}$
\[
\Vert \chi^{o}(\widehat{\theta}^{o}, \widehat{\theta}) \Vert
= 
\Vert  D_0 (\widehat{\theta}^{o} -  \widehat{\theta}) -  D_0^{-1} \nabla L^{o}(\widehat{\theta}) \Vert
\leq 
 2 \diamondsuit^{o}(r,x), 
\] 
где $\widehat{\theta}^{o}$, $\widehat{\theta}$ -- ОМП для взвешенной и невзвешенной функций правдоподобия.  
\end{theorem}

Из теорем \ref{fisher_boot} и \ref{wilks_boot} можем получить аналог теоремы \ref{dl_sq_theorem} только для взвешенной функции правдоподобия.

\begin{theorem}
\label{dl_sq_boot_theorem}

Пусть выполнено условие ($L^*$), а также условия квадратичной аппроксимации  Лапласа $(a')$. Тогда с вероятностью не менее $1 - 16 e^{-x}$ в локальной области $\Theta_1(r) \cap \Theta_2(r)$ выполнено неравенство
\[
\left| 
\sqrt{ 2\triangle L (\widehat{\theta}_1^{o}, \widehat{\theta}_2^{o}) } - 
\Vert \triangle \xi_{12}^{o} + \triangle \widehat{\theta}_{12} \Vert 
\right| \leq 
(8 + 4 f_D)  \diamondsuit^{o} (r, x).
\]
где
\[
\triangle \xi_{12}^{o}  = \Sigma (D_2^{-1} \xi_2^{o} - D_1^{-1} \xi_1^{o}),
\quad
\triangle \widehat{\theta}_{12}  = \Sigma (\widehat{\theta}_2 - \widehat{\theta}_1), 
\]
\[
\left\Vert  \Sigma D^{-1}_i  \right\Vert_{\infty} \leq f_D, 
\quad i = \{1,2\}.
\]

\end{theorem}

\begin{remark}
Для генерации нормы вектора $\xi_{12}^{o}$ можно воспользоваться статистикой $T^{o} = \sqrt{ 2\triangle L (\widehat{\theta}_1^{o}, \widehat{\theta}_1^{o} + \triangle \widehat{\theta}_{12})} $, для которой также применима теорема 
\ref{dl_sq_boot_theorem}, причем переменная $\triangle \widehat{\theta}_{12}  = 0$ для данной статистики, т.е.
\[
\left| 
T^{o} - 
\Vert \triangle \xi_{12}^{o}  \Vert 
\right| \leq 
(8 + 4 f_D)  \diamondsuit^{o} (r, x).
\]
\end{remark}

Найдем точность оценки распределения нормы вектора  $\triangle \xi_{12} + b$ при помощи бутстрепа (аналога $\triangle \xi_{12}^{o} + b^{o}$,  $b$, $b^{o}$ неслучайные), приближая их распределения к нормальному. Последовательность сравнения распределений:
\begin{enumerate}
\item $\triangle \xi_{12}^{o}(\widehat{\theta}_1, \widehat{\theta}_2) \approx \triangle \xi_{12}^{o}(\theta_1^*, \theta_2^*)$;
\item  $\Vert \triangle \xi_{12}^{o}(\theta_1^*, \theta_2^*) + b^o \Vert  \approx  \Vert \xi_{12}^{o} \Vert$, $\xi_{12}^{o} \sim \mathcal{N}(b^o, \Sigma^o) $;
\item  $\Vert \triangle \xi_{12}(\theta_1^*, \theta_2^*) + b \Vert  \approx  \Vert \xi_{12} \Vert$.
\item  $\Vert \xi_{12}^{o}  \Vert  \approx  \Vert \xi_{12} \Vert$, $\xi_{12} \sim \mathcal{N}(b, \Sigma) $;
\end{enumerate}


\noindent\textbf{1)} Рассчитаем отклонение $\triangle \xi_{12}^{o}(\widehat{\theta}_1, \widehat{\theta}_2)$ от $\triangle \xi_{12}^{o}(\theta_1^*, \theta_2^*)$.  Поскольку 
\[
\xi_i^{o}(\theta) = D_0^{-1}  \sum_{i=1}^{n} \nabla l_i(\theta) (u_i - 1),
\]
то, используя оценку для $S(\theta_2, \theta_1)$, получим  неравенство в локальной области, с вероятностью $1- 3e^{x}$, $i = \{1,2\}$
\[
\Vert \xi_i^{o}(\theta_2) - \xi_i^{o}(\theta_1) \Vert \leq 
\left(
12 \nu_0  z(x) \omega r \sigma_u \sqrt{1+x} + 
\frac{1 + \sqrt{2x}}{2n}  C(r) r c_u
\right)  
 =  2 \diamondsuit_{\xi}^{o}(r,x). 
\]
Запишем  $\triangle \xi_{12}^{o}(\theta_1, \theta_2)$ в виде 
\[
\triangle \xi_{12}^{o}(\theta_1, \theta_2) = A \left( \begin{array}{c}
\xi_1^{o}(\theta_1) \\
\xi_2^{o}(\theta_2)
\end{array}
\right),
\quad 
A = \Sigma \left( - D_1^{-1}, D_2^{-1}
\right).
\]
Тогда справедливо утверждение 
\begin{lemma}
\[
\Vert \triangle \xi_{12}^{o}(\widehat{\theta}_1, \widehat{\theta}_2) - \triangle \xi_{12}^{o}(\theta_1^*, \theta_2^*) \Vert \leq 
  4 \diamondsuit_{\xi,F}^{o}(r,x), 
\]
где $F = A^T A$, $H_2 = H_2(F) + 4p$,
\[
\diamondsuit_{\xi,F}^{o}(r,x) = \left(
12 \nu_0  z_F(x) \omega r \sigma_u \sqrt{1+x} + 
\frac{1 + \sqrt{2x}}{2n}  C(r) r c_u
\right).
\]
\end{lemma}

\noindent\textbf{2),3)} Найдем точность слабой аппроксимации $\Vert \triangle \xi_{12}^{o}(\theta_1^*, \theta_2^*) + b^o \Vert  \approx  \Vert \xi_{12}^{o} \Vert$, $\xi_{12}^{o} \sim \mathcal{N}(b^o, \Sigma^o) $. 

Будем дополнительно считать, что $\triangle \xi_{12}$, $\triangle \xi_{12}^{o}$  зависят от координаты $t$. Выявим насколько хорошо распределение  $\max_{t}\Vert \triangle \xi_{12}^{o}(\theta_1^*, \theta_2^*)(t) + b^o(t) \Vert $ приближается распределением нормы нормального вектора $\max_{t}\Vert  \xi_{12}^{o}(t)\Vert $. Норму вектора $S$ будем выражать через $\max_{\Vert\gamma \Vert = 1} \gamma^T S$, а максимум через smooth max функцию:
\[
h_\beta (x) = \beta^{-1} \log\left( \sum_{i} e^{\beta x_i} \right) .
\]
Функция $h_\beta (x)$ обладает свойством при $x \in \mathbb{R}^M$
\[
 \max_i(x_i)  \leq  h_\beta (x) \leq \max_i(x_i)  +  \frac{\log(M)}{\beta}.
\] 
\begin{lemma}
Для функции $g_{\delta}$, аппроксимирующей индикатор ($g_{\delta}$ возрастает от 0 до одного на отрезке ${\delta}$),   имеет место неравенство при $\delta = \beta^{-1} \log(M)$
\[
g_{\delta} h_\beta (x - \delta) \leq \left[\max_{0\leq i \leq M} x_i > 0 \right]  \leq g_{\delta} h_\beta (x + \delta).
\]
Следовательно, при условии $\vert \Exp g_{\delta} h_\beta(x)  -  \Exp g_{\delta} h_\beta(\widetilde{x}) \vert \leq C(\beta, M) \mu_n$
\begin{equation}
\label{max_norm_approx}
\left| \PR \left(\max_{0\leq i \leq M} x_i > 0 \right)  -  \PR  \left(\max_{0\leq i \leq M} \widetilde{x}_i > \pm 2\delta \right) \right| \leq 
C(\delta, M) \mu_n.
\end{equation}
\end{lemma}

Вынесем сдвиг $\pm 2\delta$ наружу при помощи леммы об анти-концентрации  плотности распределения $\max_{i} \widetilde{x}_i$ .
 
\begin{lemma}
Пусть $x \in \N (m, \Sigma) \in \mathbb{R}^M $,  $\sigma_1 \leq  \sqrt{\Sigma_{ii}} \leq  \sigma_2$, $a_M = \max_i (\widetilde{x}_i - m_i) /\sqrt{ \Sigma_{ii}}$, тогда $\forall c$
 \[
 \PR(\vert \max_{i} \widetilde{x}_i - c \vert \leq  \varepsilon)
 \leq \frac{4 \varepsilon}{\sigma_1} \left( \frac{\sigma_2}{\sigma_1} a_M +  \left(\frac{\sigma_2}{\sigma_1} - 1\right) \sqrt{2 \log \left(\frac{\sigma_1}{\varepsilon} \right)}+2  - \frac{\sigma_1}{\sigma_2}\right) \approx 4 \varepsilon \frac{\sigma_2}{\sigma_1^2} \sqrt{2 \log \left(  \frac{ \sigma_1 M}{\varepsilon} \right)}.
 \]
\end{lemma}
Используя предыдущую лемму и выражение \ref{max_norm_approx}, получаем соотношение для разности вероятностных мер максимумов случайных векторов, один из которых имеет нормальное распределение 
\begin{equation}
\label{max_norm_approx_1}
\left| \PR \left(\max_{0\leq i \leq M} x_i > 0 \right)  -  \PR  \left(\max_{0\leq i \leq M} \widetilde{x}_i > 0 \right) \right| \leq 
C(\delta, M) \mu_n +  \delta C_{\text{ak}}(M, \Sigma).
\end{equation}

Проведем теперь расчет верхней границы в выражении
\[
\vert \Exp g_{\delta} h_\beta(x)  -  \Exp g_{\delta} h_\beta(\widetilde{x}) \vert \leq C(\beta, M) \mu_n.
\] 

\begin{lemma}
Для функции $f = g_{\delta} h_{\beta}$, где $g_{\delta}$ имеет ограниченные производные до третьего порядка, имеет место соотношение при $\delta = \beta^{-1} \log(M)$
\[
\left| f(x + d) - f(x) - d^{T}f'(x) - d^{T}f''(x)d /2 \right |  
\leq C(\delta, M) \Vert d \Vert_{\infty}^{3},
\]
\end{lemma} 
где 
\[
C(\delta, M) =  \frac{1}{6 \delta^3 } (|g'''| + \log(M) |g''| + \log^2(M) |g''|).
\]


Пусть 
\[
\triangle \xi_{12}^{o}(\theta_1^*, \theta_2^*)(t) + b^o(t)  = \xi(t)=
\sum_{i=1}^{n} c_i(t) \varepsilon_i,
\]
где $\Var \sum_{i} c_i(t) \varepsilon_i = \sum_{i} c_i^2(t) V_i^2$. 

При помощи приближения $\varepsilon_i$ нормальными с.в. с такими же математическими ожиданиями и корреляционными матрицами получаем неравенство
\begin{equation}
\vert \Exp g_{\delta} h_\beta(\xi)  -  \Exp g_{\delta} h_\beta(\widetilde{\xi}) \vert \leq C(\delta, M) \mu_n,
\end{equation}  
где
\[
\mu_n = \sum_{i} ( \Vert \varepsilon_i \Vert_{\infty}^3 + C \log^{3/2} (M) \sigma_i^3  ) \Vert c_i \Vert_{\infty}^3 
\]


\begin{lemma}
Пусть $X$, $Y$ независимые нормальные вектора с параметрами $m_X$, $m_Y$, $\Sigma_X$, $\Sigma_Y$. 
Определим вспомогательные переменные 
\[
\triangle m = m_2 - m_1,
\quad
\triangle \Sigma = \Sigma_2 - \Sigma_1.
\] 
Справедливо неравенство
\[
| \Exp g(\delta^{-1} h(X)) - \Exp g(\delta^{-1} h(Y)) | 
\leq 
\left( 
\frac{\beta \Vert g' \Vert_\infty}{\delta} + \frac{\Vert g'' \Vert_\infty }{2 \delta^2}
\right) \Vert \triangle \Sigma \Vert_\infty + 
\frac{ \Vert g' \Vert_\infty}{2 \delta} \Vert \triangle m \Vert_\infty .
\]
\end{lemma}   



    
  \end{document}  